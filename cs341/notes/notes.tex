\ifdefined\isphone
  \documentclass[a6paper,11pt,oneside]{book}

\usepackage{algpseudocode}
\usepackage{algorithm}
\usepackage{amsfonts}
\usepackage{amsmath,amsthm,amssymb}
\usepackage{graphicx}
\usepackage{hyperref}
\usepackage{mathtools}
\usepackage{steinmetz}
\usepackage{fancyvrb}
\usepackage{textcomp}
\usepackage{gensymb}
\usepackage[normalem]{ulem}
\usepackage[T1]{fontenc}
\usepackage{cmap}

\DeclareGraphicsExtensions{.pdf,.png,.jpg}

\newcommand{\mb}[1]{\ensuremath{\mathbb{{#1}}}}
\newcommand{\setof}[1]{\ensuremath{\left \{ #1 \right \}}}
\newcommand{\tuple}[1]{\ensuremath{\left \langle #1 \right \rangle }}

\DeclareSymbolFont{bbold}{U}{bbold}{m}{n}
\DeclareSymbolFontAlphabet{\mathbbold}{bbold}

\newtheorem{statement}{Statement}
\newtheorem{rmk}{Remark}
\newtheorem{defi}{Definition}
\newtheorem{example}{Example}
\newtheorem{theorem}{Theorem}
\newtheorem{lemma}[theorem]{Lemma}
\newtheorem{proposition}[theorem]{Proposition}
\newtheorem{pf}[theorem]{Proof}
\newtheorem{corollary}[theorem]{Corollary}

% Differences
\usepackage[margin=5mm]{geometry}
\usepackage{tgheros}
\renewcommand*\familydefault{\sfdefault}

\else
  \documentclass[11pt,oneside]{book}

\usepackage{algpseudocode}
\usepackage{algorithm}
\usepackage{amsfonts}
\usepackage{amsmath,amsthm,amssymb}
\usepackage{graphicx}
\usepackage{hyperref}
\usepackage{mathtools}
\usepackage{steinmetz}
\usepackage{fancyvrb}
\usepackage{textcomp}
\usepackage{gensymb}
\usepackage[normalem]{ulem}
\usepackage[T1]{fontenc}
\usepackage{cmap}

\DeclareGraphicsExtensions{.pdf,.png,.jpg}

\newcommand{\mb}[1]{\ensuremath{\mathbb{{#1}}}}
\newcommand{\setof}[1]{\ensuremath{\left \{ #1 \right \}}}
\newcommand{\tuple}[1]{\ensuremath{\left \langle #1 \right \rangle }}

\DeclareSymbolFont{bbold}{U}{bbold}{m}{n}
\DeclareSymbolFontAlphabet{\mathbbold}{bbold}

\newtheorem{statement}{Statement}
\newtheorem{rmk}{Remark}
\newtheorem{defi}{Definition}
\newtheorem{example}{Example}
\newtheorem{theorem}{Theorem}
\newtheorem{lemma}[theorem]{Lemma}
\newtheorem{proposition}[theorem]{Proposition}
\newtheorem{corollary}[theorem]{Corollary}

% Differences
\setlength{\textheight}{9.5in}
\setlength{\textwidth}{7.0in}
\setlength{\topmargin}{-0.75in}
\setlength{\oddsidemargin}{-0.25in}
\setlength{\evensidemargin}{0.75in}
\setlength{\parskip}{0.15in}
\setlength{\parindent}{0in}

\fi

\begin{document}

% Info section

\title{CS 341 Term Notes}

\author{
    Shale Craig\\
    \texttt{sakcraig@uwaterloo.ca}
}
\maketitle

% Note: uncomment these if I add them
\tableofcontents
% \listofAlgorithms
% \listoffigures

\newpage

\subsection*{Preface} % (fold)
\label{sub:preface}
    This course was taken in Winter 2013 at the University of Waterloo under \href{https://cs.uwaterloo.ca/~tmchan/}{Timothy Chan}.

    These notes come with no guarantee of correctness, veracity, nor semblance of course materials.
    Go to class, your profs are right experts.

    Compiled with love by:
    \begin{itemize}
        \item Shale Craig - \url{http://shalecraig.com}
    \end{itemize}

    Please submit corrections, omissions, and requests for change to \url{http://github.com/shalecraig/Notes}
% subsection preface (end)

\chapter{Introduction}
    \section{Bentley's Problem}

        Find the consecutive sub-array $B$ of a given array $A$ with the maximal
        value.

        i.e: $A = [1, 2, -3, 4, -4, 0]$, $B = [1, 2, -3, 4]$

        $O(n)$-time solution exists:
        \begin{verbatim}
            max_subarray(A):
                max_ending_here = max_so_far = 0
                for x in A:
                    max_ending_here = max(0, max_ending_here + x)
                    max_so_far = max(max_so_far, max_ending_here)
                return max_so_far
        \end{verbatim}

        Basically, we keep track of the maximal subarray ending ``here''
        (including the empty array), and then compare that to the max so far.

    \section{The 3SUM problem}
        Given a set of integers $A$ and an integer $k$, find $3$ different
        numbers in $A$ that sum to $k$.

        i.e. $A = [1, 2, -3, 4, -4, 0]$, $k = 3$. $2 + 4 - 3 = k$

        $O(n^2)$-time solution exists:
        \begin{verbatim}
            sort(A);
            for i=0 to n-3 do
                a = A[i];
                j = i+1;
                l = n-1;
                while (j<l) do
                    b = A[j];
                    c = A[l];
                    if (a+b+c == k) then
                        output a, b, c;
                        exit;
                    else if (a+b+c > k) then
                        l = l - 1;
                    else
                        j = j + 1;
                    end
                end
            end
        \end{verbatim}
        Basically, we sort values in $A$, pick one value of $a$, then try to
        find values of $A[j]$ and $A[l]$ that equal $k$. We change the values
        of $l$ and $j$ according to the value of the sum compared to $k$.

    \section{Math review}
        \subsection{Asymptotic notation}
            We use $O$, $o$, $\Omega$, $\omega$, and $\Theta$ to denote the
            runtimes of different algorithms as the input sizes tend to
            $\infty$. This isn't perfect for talking about small $n$, but
            performing well for small $n$ is less an algorithms task and more an
            algebra task.

            Here's a table:
            \begin{table}[h]
                \centering
                % TODO: add asymptotic notations
                \begin{tabular}{ | l || l |}
                    \hline
                    Symbol & Condition \\ \hline
                    $f(n) \in O(g(n))$       & $|f(n)| \le k g(n)$ for some positive $k$ \\ \hline
                    $f(n) \in \Omega(g(n))$  & $|f(n)| \ge k g(n)$ for some positive $k$ \\ \hline
                    $f(n) \in \Theta(g(n))$  & $k_1 g(n) \ge |f(n)| \ge k_2 g(n)$ for some positive $k_1, k_2$ \\ \hline
                    $f(n) \in o(g(n))$       & $|f(n)| \le \epsilon g(n) $ for all positive $\epsilon$ \\ \hline
                    $f(n) \in \omega (g(n))$ & $|f(n)| \ge \epsilon g(n) $ for all positive $\epsilon$ \\ \hline
                \end{tabular}
            \end{table}

    \section{Summations}
        Summations are useful, and we only use a few in this course.

        Here they are:
        \begin{align*}
            \sum_{i=0}^n i &= \frac{n^2 + n}{2} \\
            \sum_{i=0}^n i^2 &= \frac{2n^3 + 3n^2 + n}{2} \\
            \sum_{i=0}^n a^i &= \frac{1 - a^{n-1}}{1-a} \\
            \sum_{i=0}^n \frac{1}{n} &= O(\log n + \gamma)
        \end{align*}
        % TODO: complete this list
        Note: I'm not sure this is complete.

    \section{Recurrences [Sec 4.4,4.5,4.3]}
        Recurrence relationships are often found in recursive code. We probably
        want to solve these to see the asymptotic behaviour of our algorithm.

        \subsection{The Recursion Tree Method}
            The basic idea of this approach is we break down values as they go
            down to child nodes, and determine a function for the number of
            child nodes at each level, and another the cost of each node. By
            summing these together, we get the overall cost of the function with
            respect to $n$.

            i.e.
            \begin{align*}
                f(n) &= \left\{
                    \begin{array}{lr}
                        12 f\left(\frac{n}{12}\right) + 14n & n \ge 100 \\
                        n & \text{otherwise}
                    \end{array}
                \right.
            \end{align*}
            (Drawing not provided - if you supply one, I'll add it.)

            By drawing this out, we see:
            \begin{enumerate}
                \item In level $i$, there are $12^i$ nodes.
                \item In level $i$, our cost is $14 \frac{n}{12^i}$ per node.
                \item The tree height is (approximately) $\log_{12}n$.
            \end{enumerate}

            Based on this, we get the summation:
            \begin{align*}
                f(n) &= \sum_{i=0}^{\log_{12}n} (12^i)(14 \frac{n}{12^i}) \\
                &= \sum_{i=0}^{\log_{12}n} 14 n \\
                &= 14 n \log_{12}n
            \end{align*}
            i.e we have $f(n) = 14 n \log_{12}n$

            In this example I don't count the leaf nodes, but probably should.

            TODO: evaluate the leaf nodes.
        \subsection{Master Method}
            The master method is often referred to as a ``cookbook'' or a
            ``lookup'' based method. We basically know that $f(n)$ is of a
            certain form, we can substitute it into the answer.

            Algorithms expressed in the form of
            $T(n) = a T\left( \frac{n}{b} \right) + f(n)$ can be solved using
            the master theorem. We didn't talk about other forms in class.

            We define $c = \log_b{a}$, and require that $\epsilon > 0$.

            There are three cases that we can apply it in:
            \begin{enumerate}
                \item If $f(n) \in O(n^{c-\epsilon})$, it follows that
                    $T(n) \in \Theta(n^c)$.
                \item If $f(n) \in \Theta(n^c \log^k n)$ for $k \ge 0$, it
                    follows that $T(n) \in \Theta(n^c \log^{k+1}n)$.
                \item If $f(n) \in \Omega(n^{c+\epsilon})$, it follows that
                    $T(n) \in \Theta(n^c)$.
             \end{enumerate}
        \subsection{Guess-And-Check Method}
            In this method, we guess a recursion and substitute into the
            recurrence to check for veracity.
            If our guess is incorrect we start again, using what we learned to
            re-build our guess.

            i.e.
            Guess $T(n) = c_1 n + c_2$:
            \begin{align*}
                T(n) &= T(n/2) + T(n/4) + T(n/8) + n \\
                T(n) &= c_1 n + c_2 \\
                T(n/2) + T(n/4) + T(n/8) + n &= c_1 n\frac{4+2+1}{8} + 3 c_2 + n\\
                &= \frac{7 n c_1}{8} + 3 c_2
            \end{align*}

            From this, we get that $\frac{7c_1}{8} + 1 = c_1$, so $c_1 = -8$.
            Similarily, $c_2 = 0$.

\chapter{Divide and Conquer}
    Divide and conquer algorithms break up problems into smaller sections that
    they solve recursively. They can take advantage of the fact they're solving
    smaller problems to provide a better runtime than naive solutions.

    They generally consist of three steps:
    \begin{enumerate}
        \item Break up input to smaller parts.
        \item Solve each part.
        \item Merge the solutions together.
    \end{enumerate}

    \section{The Maxima Problem}
        TODO: It was this problem.

        Given a set of points $P$, find the set of points $M$ that are a maxima
        of $P$.
        A maximal point $p \in M$ is one where there is no other point $r$ in
        $P$ where $r.x > p.x$ and $r.y > p.y$.

        Conceptually, we can break the set of points $P$ into $P_1$ and $P_2$,
        each roughly half the size of the other. Then we can solve to get $M_1$
        and $M_2$. We can merge $M_1$ and $M_2$ together, and return that.

        We need to remember to consider the base case in writing the solution.

        \begin{verbatim}
            def maxima(P):
                if len(P) <= 2:
                    return P
                sortByX(P)
                ->P1 is the 'top-left half'
                P1 = P[0 : len(P)/2]
                P2 = P[len(P)/2+1 : -1]
                M1 = maxima(P1)
                M2 = maxima(P2)
                i1 = 0
                firstM2 = M2[0]
                M = []
                while (M1[i1].y > firstM2.y):
                    M.addOne(M1[i1])
                M.addAll(M2)
                return M
        \end{verbatim}

        This algorithm takes $T(n) = 2T(n/2) + n\log n$ worst-case time. We can
        use the master method to solve this.

        By pre-sorting all the points instead of doing it in every reduction, we
        only need to do it once, so we can say that $T(n) = 2T(n/2) + n$.

        $a = 2$, $b = 2$, $f(n) = n$, $c = \log_b{a} = 1$.

        We know $f(n) \in \Theta(n^1 \log^0 n)$, so rule 2 must apply for
        $k = 0$. Thus, $T(n) \in \Theta(n^1 \log n)$.

    \section{The Closest Pair Problem (Shamos' Algorithm)}
        Given a set $P$ of points, find the distance betwwen the closest pair of
        points $\tuple{p, q}$ in $P$.

        Conceptually, we can solve this problem using a similar (but different!)
        approach than before. We will split the points into left and right
        halves and solve them independently. We will then merge them, and only
        return the best answer between the two. All that remains is to evaluate
        pairs of points that cross the separating boundary between the two
        halves, which can be done by creating a ``moving elevator'' that goes up
        to iterate through all points that are possibly valid inside this
        elevator.

        \begin{verbatim}
            def getClosestPair(Px, Py):
                if len(P) == 2
                    return dist(P[0], P[1])
                Pxl = Px[0 : len(Px)/2]
                Pxr = Px[len(Px)/2+1 : -1]
                Pyl = Py[p is in Pxl]
                Pyl = Py[p is in Pyl]
                bestLeft = getClosestPair(Pxl, Pyl)
                bestRight = getClosestPair(Pxr, Pyr)
                best = min(bestLeft, bestRight)
                windowRight = Pl[-1].x + best
                windowLeft = Pr[0].x - best
                window = Pyl[p.x > windowLeft] + Pyr[p.x < windowRight]
                for (i=0...len(window)-1):
                    k = i+1
                    while k < len(window)-1 and window[k].y-window[i].y < best:
                        if dist(window[k], window[i]) < best:
                            best = dist(window[k], window[i])
                        k = k+1
                return best
        \end{verbatim}
        We take $P_x$ and $P_y$ as input (points sorted according to x and y).
        We can express the runtime of this algorithm as
        $T(n) = 2T(n/2) + 6n \log(n) + 3$. Using the master method, we can solve
        the recurrence as $T(n) = \Theta(n \log^2 n)$.

    \section{Multiplication of Large Numbers (Karatsuba and Ofman's Algorithm)}
        This can be found at [KT, Sec 5.5 or BB, Sec 7.1]

        We want to multiply integers $A$ and $B$ (expressed as $n$ bits)
        more efficiently than the grade-school ``multiply-everything'' approach.

        Conceptually, Karatsuba used a shortcut where instead of multiplying
        $A$ with $B$ directly, he performed three multiplications of smaller
        size, which is a bit faster. If you use the algorithm recursively, it's
        asymptotically faster.

        Given integers $x$ and $y$ we want to compute $xy$. Choose a base $B$,
        and $m = \log(x)/2$
        \begin{align*}
            x &= x_1 B^m + x_2 \\
            y &= y_1 B^m + y_2 \\
            xy &= z_2 B^{2m} + z_0 + z_1 B^{m} \\
            z_2 &= x_1 y_1 \\
            z_0 &= x_0 y_0 \\
            z_1 &= x_1 y_0 + x_0 y_1 \\
            &= (x_1 + x_0)(y_1 + y_0) - z_2 - z_0
        \end{align*}

        It takes $T(n) = 3 T(n/2) + cn + d$ ($c$ and $d$ are constants) to
        calculate the value of $AB$. Solved by the master theorem, this becomes
        $T(n) \in \Theta(n^{\log_2{3}}) \approx \Theta(n^1.58)$.

    \section{Matrix Multiplication (Strassen's Algorithm)}
        This can be found at [CLRS, Sec 4.2]

        Given matrices $A$ and $B$, find the product $C$ of $A$ and $B$.

        Conceptually, Strassen realized that (like Karatsuba) you can do these
        multiplications in fewer operations that what is explicitly obvious.

        \begin{align*}
            A &= \begin{bmatrix} A_{1,1} & A_{1,2} \\ A_{2,1} & A_{2,2} \end{bmatrix} \\
            B &= \begin{bmatrix} B_{1,1} & B_{1,2} \\ B_{2,1} & B_{2,2} \end{bmatrix} \\
            C &= \begin{bmatrix} C_{1,1} & C_{1,2} \\ C_{2,1} & C_{2,2} \end{bmatrix} \\
            C_{1,1} &= A_{1,1}B_{1,1} + A_{1,2}B_{2,1} \\
            C_{1,2} &= A_{1,1}B_{1,2} + A_{1,2}B_{2,2} \\
            C_{2,1} &= A_{2,1}B_{1,1} + A_{2,2}B_{2,1} \\
            C_{2,2} &= A_{2,1}B_{1,2} + A_{2,2}B_{2,2}
        \end{align*}
        If we define an intermediate matrix $M$, we can calculate $C$ in 7
        multiplications instead of the 8 above.
        \begin{align*}
            M_1 &= (A_{1,1} + A_{2,2})(B_{1,1} + B_{2,2}) \\
            M_2 &= (A_{2,1} + A_{2,2})B_{1,1} \\
            M_3 &= A_{1,1}(B_{1,2} - B_{2,2}) \\
            M_4 &= A_{2,2}(B_{2,1} - B_{1,1}) \\
            M_5 &= (A_{1,1} + A_{1,2})B_{2,2} \\
            M_6 &= (A_{2,1} + A_{1,1})(B_{1,1} + B_{1,2}) \\
            M_7 &= (A_{1,2} + A_{2,2})(B_{2,1} + B_{2,2})
        \end{align*}
        We can now calculate $C$ in terms of $M$:
        \begin{align*}
            C_{1,1} &= M_1 + M_4 - M_5 + M_7 \\
            C_{1,2} &= M_3 + M_5 \\
            C_{2,1} &= M_2 + M_4 \\
            C_{2,2} &= M_1 + M_2 + M_3 + M_6
        \end{align*}
        Since we calculate this with $7$ multiplications of sub-problems of
        size $n/2$, this will take us $T(n) = 7 T(n/2) + in$ ($i$ is for the
        number of additions). By master method, we arrive at
        $T(n) = \Theta(n^{\log_2(7)})$ as the running time of the algorithm.

    \section{Subsection [Sec 9.2-9.3]}
        Basic problem statement: Find the $i$th element of an array $A$.
        \subsection{Randomized Quickselect}
            Basic idea is to pick a random element, seperate the elements
            into subsets that are larger and smaller than the element, then
            recurse into the subset.

            Best runtime is $O(1)$, but worst runtime is $O(n^2)$ since it is
            random. Average runtime is $O(n)$, but we can do better.

        \subsection{Blum, Floyd, Pratt, Rivest, and Tarjan's Algorithm}
            Also called ``Median of Five''.

            The way this works is pretty cool. We pick the median by finding
            the median of groups of five elements (recursive call 1). Then we
            use the knowledge of the value of the median to do a
            quickselect-style recursion (call 2) into the half that contains our
            element.

            Through this algorithm, finding the median takes $O(n)$ time
            (awesome), and the overall algorithm takes $O(n)$ time.

            Pseudocode:
            \begin{verbatim}
                select(A, k):
                    if (|A| == 1):
                        return A[0]
                    split A into groups G, each of 5 elements
                    create B = median of each g in G
                    x = select(B, |B|/2)
                    L = {a < x}
                    R = {a > x}
                    if (l <= k):
                        return select(L, k)
                    else
                        return select(R, k-|L|)
            \end{verbatim}

\chapter{Greedy Algorithms}
    Put simply, greedy algorithms solve decision problems in short amounts of
    time. They don't backtrack, and rely on a heuristic to make their decisions.
    They don't work in all circumstances, but are speedy in some. Greedy is
    (sometimes) good.

    We (almost always) need to prove greedy algorithms to be correct. Proving
    them wrong is not very hard.

    \section{Coin Changing}
        Problem definition: You are asked to make $\$1.34$ into the fewest coins
        possible. Given a dollar value and a set of coin values, what are the
        minimum coins that make up that value?

        Simple algorithm that everybody uses: pick the highest value coin that
        is less than the difference between the sum of coins you have chosen and
        the dollar value remaining.
    \section{Disjoint intervals}
        Problem definition: You are given a set $A$ of intervals where some of
        them overlap. Choose the largest subset $S$ of $A$ such that elements in
        $S$ do not overlap.

        Here are a bunch of heuristics:
        \begin{itemize}
            \item Pick the shortest range. This fails.
            \item Pick the range that conflicts with the fewest ranges. This fails.
            \item Pick the range that finishes first. This succeeds.
        \end{itemize}

    \section{Fractional knapsack}
        Problem definition: You are given a set of items $I$ and a backpack size
        $b$. Each item has a weight $i.w$ and a size $i.s$. You are allowed to
        choose fractional items. Choose backpack items such that the value of
        your bag is maximized while not being oversized.

        Correct Heuristic: While there is room in your knapsack, pick the
        remaining item that has the highest weight/size ratio, and put it in
        your knapsack. If it doesn't fit, put a fraction of it in your knapsack.

    \section{Stable marriage}
        Problem definition: You are given a set of men and women, and lists of
        their preferences for each other. Provide a ``matching'' of them such
        that no two people prefer each other over their ``match''.

        Correct Heuristic: While there are unmatched men, send an unmatched man
        to the next woman on his list that he hasn't proposed to yet. If that
        woman is already paired, she chooses between the two men, and the less
        preferred suitor becomes unmatched.

        Proof of correctness:
        \begin{itemize}
            \item Claim 1: A woman exists that he hasn't proposed to yet.

                Some women are unmatched. All women previously proposed to by
                him are matched. Therefore, some matched woman hasn't been
                proposed to by him.

            \item Claim 2: The resultant matching is stable.

                (By way of contradiction) Suppose we have matched pairs
                $\tuple{c, e}$ and $\tuple{c', e'}$ where $c$ prefers $e'$ over
                $e$ and $e'$ prefers $c$ over $c'$. Then we must have had that
                $e'$ made an offer to $c$ before $c'$. Later on, $e'$ made an
                offer to $c$, and $c$ preferred $e'$ over $e$. Since that is
                impossible, the solution must be stable.
        \end{itemize}

\chapter{Dynamic Programming}
    Basically, Dynamic Programming is solving smaller problems first before
    solving larger problems, then using the smaller problems as building blocks
    for the larger problems.

    Problems where recursive functions are called repetitively with the same
    arguments can be transformed to a dynamic programming problem. We take the
    recursive formula, and build up to it (instead of down).

    We use an array to ``cache'' results of recursive functions, and effectively
    loop our code from ``early parameters'' to higher parameters. We work
    towards our solution.

    In solutions, we need to specify:
    \begin{enumerate}
        \item Recursive formula
        \item Initial values
        \item Iteration order
        \item A sub-problem the solution is composed of
        \item The answer in terms of an array
    \end{enumerate}

    \section{Binomial coefficients}
        We want to calculate ${n \choose k}$ in the fewest number of
        calculations possible.

        Define our recursive formula:
        \begin{align*}
            C(n, k) &=
            \left\{
                \begin{array}{lr}
                    1 & n = k \text{ or } k = 0 \\
                    C(n-1, k-1) + C(n-1, k) & \text{otherwise}
                \end{array}
            \right.
        \end{align*}
        Implemented in code, this would look like:
        \begin{verbatim}
            def C(i, j):
                if (i == j || j == 0):
                    return 1
                return C(i-1, j-1) + C(i-1, j)
        \end{verbatim}
        This is a valid algorithm but in this implementation, the same values of
        $C$ will be calculated many times over and over.

        If we make $C$ an array instead of a function and determine an
        iteration order, we only need to calculate values of $C$ once.

        For the iteration order, we can iterate from $i = 0...n$, and for each of
        those we can iterate $j = 0...k$.

        We also need to determine initial values of $C$. In this case we have
        $C[i, 0] = 1$, $C[0, i] = 1$.

        The code would look like this:
        \begin{verbatim}
            def C(n, k):
                C = new array[n,k]
                for i = 0...n:
                    C[i, 0] = 1
                for i = 0...min(n, k):
                    C[0, i] = 1
                for i = 0...n:
                    for j=0...k:
                        C[i, j] = C[i-1, j-1] + C[i-1, j]
                return C[n, k]
        \end{verbatim}

        The array $C$ is $nk$ in size.
        Every element is calculated once, and takes $O(1)$ time to calculate.
        Thus, this algorithm takes $O(nk)$ time overall.

    \section{Coin Changing}
        Given a set of coins $N$ and a target value $w$, what is the minimum
        number of coins in $N$ summing to $w$.

        Sub-problem: We define $C[i, j]$ as the minimum number of coins in
        $\{N_0...N_i\}$ that sum to $j$.

        Solution: Return $C[|N|, w]$, the minimum number of coins in $N$ that
        sum to $w$.

        Base cases:
        \begin{align*}
            C[i, 0] &= 0 \\
            C[0, j] &= \infty
        \end{align*}

        Recursive formula:
        \begin{align*}
            C[i, j] &=
            \left\{
                \begin{array}{lr}
                    \min(C[i-1, j], C[i-1, j-1] + 1) & j \ge D_i \\
                    C[i-1, j] & \text{ otherwise }
                \end{array}
            \right.
        \end{align*}
        Iteration order: Iterate through all numbers $i = 1...n$, for each of
        those iterate through $j = 1...w$.

        Code:
        \begin{verbatim}
            def minGrouping(D, w):
                for i = 0...n:
                    C[i, 0] = 0
                for j = 1...w:
                    C[0, j] = infinity
                for i = 1...n:
                    for j=1...w:
                        if j < D[i]:
                            C[i, j] = min(C[i-1, j], C[i-1, j-1] + 1)
                        else:
                            C[i, j] = C[i-1, j]
                return C[|D|, w]
        \end{verbatim}
        The array $C$ takes $O(w|N|)$ space. Each element in $C$ is calculated
        exactly once and takes $O(1)$ time to calculate, so the entire algorithm
        takes $O(w|N|)$ time total.

    \section{0-1 Knapsack}
        Problem definition: Given a set of $n$ items with values $V$ and weights
        $W$, maximize the total value without the weight going over $w$.

        Subproblem definition: $C[i, j]$ is the highest value possible with the
        first $i$ elements, at a weight limit of $j$.

        Answer: $C[n, w]$

        Base cases: $C[0, j] = C[i, 0] = 0$

        Recursive formula:
        \begin{align*}
            C[i, j] &=
            \left\{
                \begin{array}{lr}
                    \max(C[i-1, j], C[i-1, j-W_i] + V_i) & j \ge w_i \\
                    C[i-1, j] & \text{ otherwise }
                \end{array}
            \right.
        \end{align*}

        Code:
        \begin{verbatim}
            def C(V, W, w):
                n = |V|
                for i=0...n:
                    C[i, 0] = 0
                for j=0...w:
                    C[0, j] = 0
                for i=1...n:
                    for j=1...w:
                        if (j >= W[i]):
                            C[i, j] = max(C[i-1, j], C[i-1, j-W[i]] + V[i])
                        else:
                            C[i, j] = C[i-1, j]
                return C[n, w]
        \end{verbatim}
    \section{Longest common subsequence}
        Reference for this can be found at [Sec 15.4]

        Problem statement: Given strings $A$ and $B$, return the longest
        subsequence of characters that appear in both strings.

        i.e. $A$ = ``ALGORITHM'', $B$ = ``LOGARITHM''.

        The longest common substring is ``LORITHM''.

        Subproblem: $C[i, j]$ is the length of the longest substring of the
        first $i$ characters of $A$ and the first $j$ characters of $B$.

        Answer: $C[m, n]$, where $m = |A|$, $n = |B|$.

        Base case: $C[i, 0] = C[0, j] = 0$  ($i = 0...m$, $j = 0...n$).

        Recursive formula:
        \begin{align*}
            C[i, j] &=
            \left\{
                \begin{array}{lr}
                    \max(C[i-1, j], C[i, j-1]) & A_i \ne B_j \\
                    \max(C[i-1, j], C[i, j-1], C[i-1, j-1]+1) & A_i = B_j \\
                \end{array}
            \right.
        \end{align*}
        i.e. If the characters at $i$ and $j$ don't match, use the max of
        $\tuple{i-1, j}$ and $\tuple{i, j-1}$.
        If they do match, use the max of $\tuple{i-1, j}$, $\tuple{i, j-1}$, and
        $\tuple{i-1, j-1}+1$.

        Code:
        \begin{verbatim}
            def LCS(A, B):
                n = |A|
                m = |B|
                C = []
                for i = 0...m:
                    C[i, 0] = 0
                for j = 0...n:
                    C[0, j] = 0
                for i = 0...m:
                    for j = 0...n:
                        if (A[i] == B[j]):
                            C[i, j] = max(C[i-1, j], C[i, j-1], C[i-1, j-1]+1)
                        else:
                            C[i, j] = max(C[i-1, j], C[i, j-1])
                return C[n, m]
        \end{verbatim}

    \section{Minimum-length triangulation}
        Reference for this can be found at [CLR (1st ed.), Sec 16.4]

        Given a convex polygon $P$ with $n$ vertices, find the triangulation
        with minimum length of chords.

        Brute force takes $\Omega\left(\frac{4^n}{n^{3/2}}\right)$ possible
        triangulations. DP does better than this.

        Sub-problems: ($1 \le i \le j \le n$)
        \begin{align*}
            C[i, j] = \text{ length of the minimum-length triangulation of
                the sub-polygon with the vertices running from $i$ to $j$ (then looping back) }
        \end{align*}

        Answer: $C[1, n]$

        Base-cases: $C[i, i+1] = C[i, i+2] = 0$

        Recursive formula:
        \begin{align*}
            C[i, j] &= \min_{k \in \{ i+1 \ldots j-1 \}} \left\{
                C[i, k] + C[k, j] + d(V_i, V_k) + d(V_j, V_k)
            \right.
        \end{align*}
        i.e. $C[i, j]$ is the minimum length for any way you split the polygon,
        which is the cost of the left half plus the right half plus the distance
        of the halves.

        Iteration order was not written down, but I'm sure it was complicated.

\chapter{Graph Algorithms}
    \section{BFS/DFS}
        Skipped writing this one, it's in my notes, and seems to be pretty
        trivial. The only thing to watch out for is to mark vertices as
        discovered because ``this is a graph'' and not a tree.

    \section{Connectedness}
        Skipped this class?
    \section{Cycle Detection}
        Skipped this class?
    \section{2-Coloring}
        Skipped this class?
    \section{Topological Sorting}
        Problem description: For a directed graph $G=(V, E)$, return a vertex
        order such that for all edges, the ``from vertex'' appears before the
        ``to vertex''.

        Conceptually, we visit all vertices using BFS. If we reach any vertex
        we have already touched, then we know there is a cycle. Otherwise, we
        are able to put elements into L in the order they are explored.
        \begin{verbatim}
            def topologicalSort(V, E):
                L = []
                while there are unmarked vertices:
                    n = an unmarked vertex
                    visit(n)
                return L
            def visit(n):
                if n has a temporary mark, we have found a cycle
                if n is not marked:
                    mark n temporarily
                    for each node m with an edge from n to m:
                        visit(m)
                    mark n permanently
                    add n to L
        \end{verbatim}
    \section{Strongly Connected Components (Kosaraju and Sharir's Algorithm)}
        Given a directed graph $G = (V, E)$, partion $V$ into components such
        that for all $u, v$ in some component, there exists a path from $u$ to
        $v$ and there exists a path from $v$ to $u$.

        i.e. we want to simplify/condense a directed graph into a DAG.

        \begin{enumerate}
            \item Run DFS(G), number vertices in the order that they finish.
            \item Form $G^T$ (the transpose of $G$ - arrows are reversed).
            \item Run DFS($G^T$), preferring higher-numbered vertices.
            \item Return the vertices from each DFS tree as components.
        \end{enumerate}
        Proof:
        Take a DFS tree $T$ of $G^T$. Let $r$ be a root, $u$ be any
        Vertex of $T$.
        \begin{enumerate}
            \item $r$ has a higher number than $u$ (if not, pick $u$ first).
            \item There exists a path $u \to r$ in $G$, since there exists a
            path $r \to u$ in $G^T$.
            \item There exists a path $r \to u$ in $G$.

                By way of contradiction, assume that there is no such path:

                    If $u$ was discovered first, then $r$ finished before $u$,
                        which means they would've been ordered differently.

                    If $r$ was discovered first, $r \not \to u$ by assumption,
                    $r$ finished before $u$.
            \item For all $u, v$ in $T$, there exists a path
            $u \leftrightarrow v$, since there exist paths $u \leftrightarrow r$
            and $r \leftrightarrow v$.
            \item There exists a path $u \leftrightarrow v$ in $G$ implies
            $\tuple{u, v}$ are in the same tree.
        \end{enumerate}
    \section{Minimum Spanning Trees}
        A spanning tree is a subgraph of a graph that is a tree and connects to
        all vertices.

        A minimum spanning tree is a subgraph of a graph that has the least (or
        equal to the lowest) sum of weighted edges in the spanning tree compared
        to all other spanning trees of the graph.

        \subsection{Kruskal's Algorithm}
            Kruskal's Algorithm is a greedy algorithm to find the minimum
            spanning tree of a graph.

            Basic idea of algorithm is as follows:
            \begin{itemize}
                \item Create a set of trees $F$ from all vertices in $V$.
                \item Create a set of edges $S$ from all edges in $E$.
                \item While $S$ is nonempty and $F$ is not spanning:
                    \begin{itemize}
                        \item Remove an edge $e$ with minimum weight from $S$.
                        \item If $e$ connects two trees in $F$, remove the two
                        trees from $F$, join them by $e$ and add that to $F$.
                    \end{itemize}
                \item Return $F$.
            \end{itemize}
        \subsection{Prim's Algorithm}
            Prim's Algorithm is a greedy algorithm to find the minimum
            spanning tree of a graph.

            Input: A connected weighted graph $G = (V, E)$.
            \begin{itemize}
                \item Initialize $V_{seen} = \{x\}$ ($x$) is an arbitrary node.
                \item Initialize $E_{mst} = \{\}$.
                \item While $V_{seen} != V$:
                    \begin{itemize}
                        \item Choose an edge $\tuple{u,v}$ with minimal weight
                        so $u$ is in $V_{seen}$ and $v$ is not.
                        \item Add $v$ to $V_{seen}$ and $\tuple{u, v}$
                        to $E_{mst}$.
                    \end{itemize}

                \item Return $E_{mst}$.
            \end{itemize}

            If we set $n = |V|$ and $m = |E|$, we can get it to be $O(n^2 + m)$
            using no fast data structures, $O(m \log n)$ using a heap, and
            $O(n \log n + m)$ using a Fibonacci Heap.

    \section{Shortest Paths}
        Given a weighted directed graph $G = (V, E)$, find a path $P$ from $s$
        to $t$ such that we minimize the ``weight'' of the path.
        \subsection{DP for DAG case}
            We can solve this problem using DP.

            Sub-Problem: $\delta[v] = w($ shortest path $s \to v)$.

            Answer: $\delta[t]$

            Base case: $\delta[s] = 0$

            Recursive formula:
            \begin{align*}
                \delta[i] &= \min_{u \in V : \tuple{u,v} \in E} \{ \delta[u] + w(u,v)\}
            \end{align*}

            Iteration order: Topological sort order.

            Analysis: $O(n + \sum_{v \in V} \text{inDeg}(v)) = O(m + n)$ time.
        \subsection{Dijkstra's Algorithm}
            Dijkstra's Algorithm assumes a positive weight for all edges in the
            graph.

            We compute d[v] = shortest path weight from s to v
            \begin{verbatim}
                // Assumes positive weight
                S = {s}
                delta[s] = 0
                while S != V:
                    pick edge(u,v) with u in S, v in V-S that minimizes delta[u] + w(u, v)
                    insert v to S
                    delta[v] = delta[u] + w(u, v)
            \end{verbatim}

            In fact, Dijkstra's algorithm is similar to Prim's algorithm, but
            the selection criteria changes from pick the shortest ``edge'' from
            a $u$ to a $v$ to pick the shortest ``edge + delta[v]''.

            It runs in $O(n \log n + m)$ time, where $m$ is the number of edges.

            \uline{Correctness:}
            Claim: If $\tuple{u, v}$ is the edge with the shortest
            $delta[u]+w(u,v)$, then the shortest weight from $s$ to $v$ is equal
            to $delta[u]+w(u,v)$.

            Proof: There is a path from $s$ to $v$ of weight $delta[u] + w(u,v)$
            Consider another path $P$ from $s$ to $v$.
            P contains edge $\tuple{u',v'}$ from $S$ to $S-V$.
            Then $w(p) \ge delta[u'] + w(u', v') + 0 \ge delta[u] + w(u,v)$.

        \subsection{All-Pairs Shortest Paths}
            Problem statement: For every pair of vertices $\tuple{u,v}$, find
            the shortest path $u$ to $v$.
            \subsubsection{Dijkstra Repeatedly}
                Runs in $O(n(n \log n + m))$, but assumes all edges are
                positive.
            \subsubsection{DP \#1}
                Subproblems $D[i,j,k]$ is the minimum weight over all paths
                from $i$ to $j$ of length $\le k$.

                Answers: $D[i, j, n-1]$ for all $i$, $j$

                Base Case:
                \begin{align*}
                    D[i, j, 0] &=
                    \left\{
                        \begin{array}{lr}
                            0 & l = j \\
                            \infty & \text{ otherwise }
                        \end{array}
                    \right.
                \end{align*}

                Recursive Formula:
                \begin{align*}
                    D[i, j, k] &= \min_{\ell \in \{ 1 \ldots n \}} D[i, \ell, k-1] + w(i, j)
                \end{align*}

                Evaluate in increasing $k$.
                \begin{verbatim}
                    for i = 1...n
                        for j = 1...n
                            for k = 0...n-1
                                ...
                \end{verbatim}
                There are $\Theta(n^3)$ entries, each one takes $\Theta(n)$
                time. Overall, it is $\Theta(n^4)$ time.

            \subsubsection{DP \# 2}
                Same setup as DP \# 1, but only $k = \{1, 2, 4, 8, ... (n-1) \text{'s next power of 2} \}$

                We choose the middle vertex $\ell$, then join things together:

                Recursive formula:
                \begin{align*}
                    D[i, j, k] &= \min_{\ell \in \{ 1 \ldots n \}} \{ D[i, \ell, k/2] + D[\ell, j, k/2] \}
                \end{align*}

                There are $\Theta(n^2 \log n)$ entries, each one takes
                $\Theta(n)$ time to compute. Overall, $\Theta(n^3 \log n)$ time.
            \subsubsection{Floyd-Warshall Algorithm}
                This is another DP variant.

                Subproblems: $D[i, j, k]$ minimum weight over all paths $i$ to
                $j$ with only intermediate vertices in $\{ 1 \ldots k \}$.

                Answers: $D[i, j, n]$ for any vertices $\tuple{i, j}$.

                Base Case:
                \begin{align*}
                    D[i, j, 0] &=
                        \left\{
                            \begin{array}{lr}
                                w(i, j) & i \ne j \\
                                0 & i = j
                            \end{array}
                        \right.
                \end{align*}
                Recursive formula:
                \begin{align*}
                    D[i, j, k] &= \min \{ D[i, j, k-1], D[i, k, k-1] + D[k, j, k-1] \}
                \end{align*}
                i.e. the minimum of the path that doesn't use $k$ and the path
                from $i$ to $k$ plus the minimal path $k$ to $j$.

                The iteration order will be weird (increasing $k$). I don't have
                it written down, but would need to think about it.

                Every edge takes $O(1)$ time to compute, there are $O(n^3)$
                edges. I suppose this takes $O(n^3)$ time overall.

\chapter{NP/P/etc}
    \section{Theory of NP-Completeness}
        All I have written down is that proving problems is hard.

        I'll describe NP-Completness later in the NP-Completness section.
    \section{Problem Specifications}
        \subsection{Decision Problem}
            Decision Problems are problems that output ``yes/no'' answers.
        \subsection{Search Problems}
            Search Problems are problems that determine the existence of at
            least one solution to a given problem.
    \section{Definition of P}
        Class $P$ is the class of ``easy''/``tractable'' problems.

        $P$ is the set of problems that are solveable in worst-case polynomial
        time i.e. $O(n^d)$ for some constant $d$. We require that $n$ is a
        reasonable representation of the input size, measured in bits.
    \section{Definition of NP}
        Class $NP$ is the class of ``hard''/``intractable'' problems.

        $NP$ is the set of problems that have solutions that are verifiable in
        $O(n^d)$ time for some constant $d$. We require that $n$ is a reasonable
        representation of the input size, measured in bits.

        All decision problems that can be expressed in the form:
        \begin{itemize}
            \item \uline{Intput} $\tuple{x}$
            \item \uline{Output} ``yes'' iff there exists an object $y$ such
                that property $R(x, y)$ holds.

                Where:
                \begin{enumerate}
                    \item $y$ has polynomial size
                    \item $R(x, y)$ can be verified in polynomial time.
                \end{enumerate}
                We call $y$ the certificate, and evaluating $R$ verification.

                NP $\leftrightarrow$ ``Non-Deterministic Polynomial''
        \end{itemize}

    \section{P vs NP vs EXPTIME}
        \begin{align*}
            P \subseteq NP \subseteq EXPTIME
        \end{align*}
        EXPTIME is the set of problems solveable in $O(2^{n^d})$ time.

        Proving (or disproving) that $P = NP$ is a million dollar proof.
    \section{Polynomial-time reductions}
        To prove that problems are equivalently hard, we want to
        provide polynomial-time ``reductions'' from one type of problem to
        another.

        We can say that $L_1$ polynomial-time reduces to $L_2$ if arbitrary
        instances of problem $L_1$ can be solved using polynomial time
        computational steps, plus polynomial calls to an oracle that solves
        $L_2$.

        We say that $L_1 \le_p L_2$ if there is a polynomial time function $f$
        so that $L_1$ on $x$ is a ``yes'' iff output of $L_2$ on $f(x)$ is
        ``yes''.
    \section{NP-Completeness}
        NP-Complete problems are a set of NP problems that all polynomial-reduce
        to each other. The implications of proving that one of these problems is
        in P would mean that $P = NP$.

        The requirement for NP-Completeness seems pretty trivial, which makes it
        easier for us to prove.
        \begin{align*}
            L &\in NP \\
            L_0 \le_p L
        \end{align*}
        If $L_0$ is known to be NP-Complete, then $L$ is NP-Complete too.
        \subsection{Proving NP-Completeness}
    \section{SAT (Cook-Levin theorem)}
    \section{SAT to 3SAT}
    \section{3SAT to independent set/vertex cover/clique}
    \section{Vertex cover to Hamiltonian cycle/traveling salesman}
    \section{Vertex cover to subset sum/knapsack}
    \section{Beyond NP}
        \subsection{Halting Problem}
        \subsection{Turing's theorem}

\end{document}
